\documentclass[12pt,a4paper]{article}
\usepackage[margin=2cm]{geometry}
\usepackage{amsmath,amsfonts,amsthm}
\title{Notes on the Mittag--Leffler function}
\author{William McLean}
\date{\today}
%%%%%%%%%%%%%%%%%%%%%%%%%%%%%%%%%%%%%%%%%%%%%%%%%%%%%%%%%%%%%%%%%%%%%%%%%%%%%%%
\newtheorem{theorem}{Theorem}
\newtheorem{lemma}[theorem]{Lemma}
%%%%%%%%%%%%%%%%%%%%%%%%%%%%%%%%%%%%%%%%%%%%%%%%%%%%%%%%%%%%%%%%%%%%%%%%%%%%%%%
\DeclareMathOperator*{\res}{res}
%%%%%%%%%%%%%%%%%%%%%%%%%%%%%%%%%%%%%%%%%%%%%%%%%%%%%%%%%%%%%%%%%%%%%%%%%%%%%%%
\begin{document}
\maketitle
%%%%%%%%%%%%%%%%%%%%%%%%%%%%%%%%%%%%%%%%%%%%%%%%%%%%%%%%%%%%%%%%%%%%%%%%%%%%%%%
The two-argument, Mittag--Leffler function is defined by the power series
\[
E_{\alpha,\beta}(z)=\sum_{n=0}^\infty\frac{z^n}{\Gamma(\beta+n\alpha)}
\quad\text{for $z\in\mathbb{C}$, $\alpha>0$, $\beta>0$.}
\]
The reciprocal of the Gamma function may be written as an integral along a
Hankel contour,
\[
\frac{1}{\Gamma(\beta+n\alpha)}=\frac{1}{2\pi i}\int_{-\infty}^{0^+}
    \frac{e^w\,dw}{w^{\beta+n\alpha}},
\]
and therefore, if $|z|<|w^\alpha|$ for all~$w$ on the contour, then
\[
E_{\alpha,\beta}(z)=\frac{1}{2\pi i}\int_{-\infty}^{0^+}\frac{e^w}{w^\beta}
    \sum_{n=0}^\infty(zw^{-\alpha})^n\,dw
    =\frac{1}{2\pi i}\int_{-\infty}^{0^+}\frac{e^w}{w^\beta}\,
    \frac{dw}{1-zw^{-\alpha}},
\]
and so
\begin{equation}\label{eq: integral repn}
E_{\alpha,\beta}(z)=\frac{1}{2\pi i}\int_{-\infty}^{0^+}
    \frac{e^w\,dw}{w^\beta-zw^{\beta-\alpha}}.
\end{equation}


%%%%%%%%%%%%%%%%%%%%%%%%%%%%%%%%%%%%%%%%%%%%%%%%%%%%%%%%%%%%%%%%%%%%%%%%%%%%%%%
\section{Asymptotic expansion}
Since
\[
\frac{1}{w^\beta-zw^{\beta-\alpha}}=\frac{-1}{zw^{\beta-\alpha}}
    \,\frac{1}{1-w^\alpha z^{-1}}
\]
and
\[
\frac{1}{1-w^\alpha z^{-1}}=\sum_{n=0}^N(w^\alpha z^{-1})^n
    +\frac{(w^\alpha z^{-1})^{N+1}}{1-w^\alpha z^{-1}},
\]
we see from~\eqref{eq: integral repn} that
\[
E_{\alpha,\beta}(z)=\sum_{n=0}^N\frac{-1}{2\pi i}\int_{-\infty}^{0^+}
    \frac{e^w(w^\alpha z^{-1})^n}{zw^{\beta-\alpha}}\,dw+R_N(z),
\]
where the remainder term is
\[
R_N(z)=\frac{-1}{2\pi i}\int_{-\infty}^{0^+}
    \frac{e^w(w^\alpha z^{-1})^{N+1}\,dw}{zw^{\beta-\alpha}(1-w^\alpha z^{-1})}.
\]
The $n$th term equals
\[
\frac{-z^{-1-n}}{2\pi i}\int_{-\infty}^{0^+}e^w w^{(n+1)\alpha-\beta}\,dw,
\]
and if we assume $\alpha-\beta>-1$ then by collapsing the contour onto the 
negative real axis and using the substitutions~$w=re^{\pm i\pi}$, we obtain
\[
\frac{-1}{2\pi i}\int_{-\infty}^{0^+}e^w w^{(n+1)\alpha-\beta}\,dw
    =\frac{e^{i\pi[(n+1)\alpha-\beta]}-e^{-i\pi[(n+1)\alpha-\beta]}}{2\pi i}
    \int_0^\infty e^{-r}r^{(n+1)\alpha-\beta}\,dr
\]
so
\[
E_{\alpha,\beta}(z)=R_N(z)+\frac{1}{\pi}\sum_{n=0}^N
    \sin\pi[(n+1)\alpha-\beta]\,\Gamma\bigl((n+1)\alpha-\beta+1\bigr)z^{-1-n}.
\]
Also,
\[
R_n(z)=\frac{z^{-1-N}}{2\pi i}\int_{-\infty}^{0^+}
    \frac{e^w w^{(N+2)\alpha-\beta}}{w^\alpha-z}\,dw
\]
If $z=x>0$, then this integrand has a pole at~$w=x^{1/\alpha}$ (in addition to
the branch point at~$w=0$) so we collect a residue when collapsing the contour 
onto the negative real axis.  Since
\[
\res_{w\to x^{1/\alpha}}\frac{e^w w^{(N+2)\alpha-\beta}}{w^\alpha-x}
    =\exp(x^{1/\alpha})\,x^{(N+2)-\beta/\alpha}\lim_{w\to x^{1/\alpha}}
    \frac{w-x^{1/\alpha}}{w^\alpha-x},
\]
and the limit equals $x^{-1+1/\alpha}/\alpha$, we find that
\[
R_N(x)=\frac{x^{(1-\beta)/\alpha}}{\alpha}\,\exp(x^{1/\alpha})
    +\frac{x^{-1-N}}{2\pi i}\int_0^\infty e^{-r}r^{(N+2)\alpha-\beta}\biggr(
     \frac{e^{-i\pi[(N+2)\alpha-\beta]}}{r^\alpha e^{-i\pi\alpha}-x}
    -\frac{e^{i\pi[(N+2)\alpha-\beta]}}{r^\alpha e^{i\pi\alpha}-x}\biggr)\,dr.
\]
Here,
\[
\frac{1}{2i}\biggr(
     \frac{e^{-i\pi[(N+2)\alpha-\beta]}}{r^\alpha e^{-i\pi\alpha}-x}
    -\frac{e^{i\pi[(N+2)\alpha-\beta]}}{r^\alpha e^{i\pi\alpha}-x}\biggr)
    =\frac{r^\alpha\sin\pi[\beta-(N+1)\alpha]+x\sin\pi\alpha}%
{(r^\alpha-x\cos\pi\alpha)^2+x^2\sin^2\pi\alpha},
\] 
so we have
\[
E_{\alpha,\beta}(x)=\frac{x^{(1-\beta)/\alpha}}{\alpha}\,\exp(x^{1/\alpha})
    +\frac{1}{\pi}\sum_{n=0}^N
    \sin\pi[(n+1)\alpha-\beta]\,\Gamma\bigl((n+1)\alpha-\beta+1\bigr)x^{-1-n}
    +\tilde R_N(x),
\]
where
\[
\tilde R_N(x)=\frac{x^{-1-N}}{\pi}\int_0^\infty e^{-r}r^{(N+2)\alpha-\beta}
    \frac{r^\alpha\sin\pi[\beta-(N+1)\alpha]+x\sin\pi\alpha}%
{(r^\alpha-x\cos\pi\alpha)^2+x^2\sin^2\pi\alpha}\,dr.
\]

If $z=-x$ for~$x>0$, then we obtain in the same way
\[
R_N(-x)=(-1)^{N+1}\,\frac{x^{-1-N}}{\pi}\int_0^\infty 
    e^{-r}r^{(N+2)\alpha-\beta}
    \frac{r^\alpha\sin\pi[\beta-(N+1)\alpha]-x\sin\pi\alpha}%
{(r^\alpha+x\cos\pi\alpha)^2+x^2\sin^2\pi\alpha}\,dr.
\]
Let
\begin{equation}\label{eq: Cpm}
C_\pm(\alpha, \beta, N)=\sup_{0\le y<\infty}
    \biggl|\frac{ay\pm b}{(y\mp c)^2+b^2}\biggr|
\end{equation}
where
\begin{equation}\label{eq: A B C}
a=\sin\pi[\beta-(N+2)\alpha],\qquad b=\sin\pi\alpha,\qquad c=\cos\pi\alpha,
\end{equation}
and note that
\[
\int_0^\infty e^{-r}r^{(N+2)\alpha-\beta}\,dr
    =\Gamma\bigl((N+2)\alpha-\beta+1\bigr).
\]
Thus,
\[
|\tilde R_N(x)|\le\frac{C_+(\alpha,\beta,N)}{\pi}\,
    \Gamma\bigl((N+2)\alpha-\beta+1\bigr)x^{-2-N}
\]
and
\[
|R_N(-x)|\le\frac{C_-(\alpha,\beta,N)}{\pi}\,
    \Gamma\bigl((N+2)\alpha-\beta+1\bigr)x^{-2-N}.
\]

\begin{lemma}
Consider $C_\pm(\alpha,\beta,N)$ defined by \eqref{eq: Cpm}~and 
\eqref{eq: A B C}.
\begin{enumerate}
\item
If $a=0$ and $c\ge0$, then
\[
C_+(\alpha, \beta, N)=\frac{1}{|b|}\quad\text{and}\quad
C_-(\alpha, \beta, N)=|b|.
\]
\item 
If $a=0$ and $c<0$, then
\[
C_+(\alpha, \beta, N)=|b|\quad\text{and}\quad
C_-(\alpha, \beta, N)=\frac{1}{|b|}.
\]
\item
Otherwise, if $a\ne0$, then we let $S_\pm$ denote the set of positive (real) 
roots of the quadratic equation
\begin{equation}\label{eq: A B C quadratic}
ay^2\pm2by-(a+2bc)=0.
\end{equation}
\begin{enumerate}
\item If $S=\emptyset$, then
\[
C_+(\alpha, \beta, N)=|b|\quad\text{and}\quad C_-(\alpha, \beta, N)=|b|.
\]
\item If $S\ne\emptyset$, then
\[
C_\pm(\alpha,\beta,N)=\max\biggl(|b|,
\max_{y\in S_\pm}\biggl|\frac{ay\pm b}{(y\mp c)^2+b^2}\biggr|\biggr).
\]
\end{enumerate}
\end{enumerate}
\end{lemma}
\begin{proof}
Let
\[
f_\pm(y)=\frac{ay\pm b}{(y\mp c)^2+b^2},
\]
and observe that $f_\pm(0)=B$ because $b^2+c^2=1$, and that $f_\pm(y)\to0$ 
as~$y\to\infty$.  If $a=0$~and $c\ge0$, then $|f_+(y)|\le|f_+(c)|=|b|^{-1}$
for~$0\le y<\infty$, whereas $|f_-(y)|\le|f_-(0)|=|b|$.  If $a=0$~and $c<0$,
then $|f_+(y)|\le|f_+(0)|=|b|$~and $|f_-(y)|\le|f_-(c)|=|b|^{-1}$ instead.

It remains to deal with the case~$a\ne0$.  We see that $f_\pm'(y)=0$ if and 
only if
\[
a[(y\mp c)^2+b^2]=2(ay\pm b)(y\mp c),
\]
which simplifies to give the quadratic equation~\eqref{eq: A B C quadratic}.
The maximum value of~$|f_\pm(y)|$ must occur either at~$y=0$ or at an element 
of~$S_\pm$.
\end{proof}


%%%%%%%%%%%%%%%%%%%%%%%%%%%%%%%%%%%%%%%%%%%%%%%%%%%%%%%%%%%%%%%%%%%%%%%%%%%%%%%
\end{document}

